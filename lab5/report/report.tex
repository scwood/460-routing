\documentclass[fleqn,11pt]{article}

\usepackage[letterpaper,margin=0.75in]{geometry}

\usepackage{amsmath}
\usepackage{booktabs}
\usepackage{graphicx}
\usepackage{listings}
\usepackage{verbatim}

\setlength{\parindent}{1.4em}

\begin{document}

\lstset{
  language=Python,
  basicstyle=\small,          % print whole listing small
  keywordstyle=\bfseries,
  identifierstyle=,           % nothing happens
  commentstyle=,              % white comments
  stringstyle=\ttfamily,      % typewriter type for strings
  showstringspaces=false,     % no special string spaces
  numbers=left,
  numberstyle=\tiny,
  numbersep=5pt,
  frame=tb,
}

\title{Lab 5 Report}

\author{Kyle Cornelison, Spencer Wood}

\date{4/18/2017}

\maketitle

\section{Routing Algorithm Details}
  Our routing algorithm is based on the distance-vector routing algorithm which
  uses the Bellman-Ford algorithm used to find the shortest path from each node
  to all of the other nodes within a graph. Our implementation relies on nodes
  broadcasting their distance-vector to each of their neighbors, defined to be
  one hop away. Upon receiving a broadcasted distane-vector packet the node will
  compare its distance vector to the one it received and update any costs, and
  forwarding entries accordingly.

\section{Experiment 1 - Line of 5 Nodes}
  \subsection{Network Configuration}
    \verbatiminput{../networks/five-nodes-line.txt}
  \subsection{Experiment Description}
    The following experiment utilizes our routing algorithm to determine routes
    between the nodes in the network displayed above. We then send a packet
    from node 1 to node 5 and trace the path it takes. The path taken should be
    in sequence from node 1 to node 5 (1, 2, 3, 4, 5).

  \subsection{Experiment Results}
  \subsubsection{Distance Vectors}
    \begin{verbatim}
      To: n1, Packet #20
      {
        "n1": {
          "cost": 0,
          "address": null
        },
        "n2": {
          "cost": 1,
          "address": 2
        },
        "n3": {
          "cost": 2,
          "address": 4
        },
        "n4": {
          "cost": 3,
          "address": 6
        },
        "n5": {
          "cost": 4,
          "address": 8
        }
      }

      To: n2, Packet #24
      {
        "n1": {
          "cost": 1,
          "address": 1
        },
        "n2": {
          "cost": 0,
          "address": null
        },
        "n3": {
          "cost": 1,
          "address": 4
        },
        "n4": {
          "cost": 2,
          "address": 6
        },
        "n5": {
          "cost": 3,
          "address": 8
        }
      }

      To: n3, Packet #20
      {
        "n1": {
          "cost": 2,
          "address": 1
        },
        "n2": {
          "cost": 1,
          "address": 3
        },
        "n3": {
          "cost": 0,
          "address": null
        },
        "n4": {
          "cost": 1,
          "address": 6
        },
        "n5": {
          "cost": 2,
          "address": 8
        }
      }

      To: n4, Packet #25
      {
        "n1": {
          "cost": 3,
          "address": 1
        },
        "n2": {
          "cost": 2,
          "address": 3
        },
        "n3": {
          "cost": 1,
          "address": 5
        },
        "n4": {
          "cost": 0,
          "address": null
        },
        "n5": {
          "cost": 1,
          "address": 8
        }
      }

      To: n5, Packet #21
      {
        "n1": {
          "cost": 4,
          "address": 1
        },
        "n2": {
          "cost": 3,
          "address": 3
        },
        "n3": {
          "cost": 2,
          "address": 5
        },
        "n4": {
          "cost": 1,
          "address": 7
        },
        "n5": {
          "cost": 0,
          "address": null
        }
      }
    \end{verbatim}
  \subsubsection{Trace}
    \begin{verbatim}
      1000.0 n1 forwarding packet to 8
      1000.0018 n2 forwarding packet to 8
      1000.0036 n3 forwarding packet to 8
      1000.0054 n4 forwarding packet to 8
      1000.0072 n5 received packet
    \end{verbatim}

\section{Experiment 2 - Ring of 5 Nodes}
  \subsection{Network Configuration}
    \verbatiminput{../networks/five-nodes-ring.txt}
  \subsection{Experiment Description}
    The following experiment utilizes our routing algorithm to determine routes
    between the nodes in the network displayed above. We then send a packet
    from node 1 to node 5 and trace the path it takes. The path taken should be
    in sequence from node 1 to node 5 (1, 5). We then take down a link
    (the link between 1 and 5) and trace the new path taken from node 1 to
    node 5 (1, 2, 3, 4, 5).

  \subsection{Experiment Results - No Failure}
  \subsubsection{Distance Vectors}
    \begin{verbatim}
    \end{verbatim}
  \subsubsection{Trace}
    \begin{verbatim}
    \end{verbatim}

  \subsection{Experiment Results - With Failure}
  \subsubsection{Distance Vectors}
    \begin{verbatim}
    \end{verbatim}
  \subsubsection{Trace}
    \begin{verbatim}
    \end{verbatim}

\section{Experiment 3 - Mesh Network}
  \subsection{Network Configuration}
    \verbatiminput{../networks/fifteen-nodes.txt}
  \subsection{Experiment Description}
    The following experiment utilizes our routing algorithm to determine routes
    between the nodes in the network displayed above. We then send a packet
    from node 1 to node 5 and trace the path it takes. The path taken should be
    in sequence from node 1 to node 5 (1, 4, 5). We then take down a link
    (the link between 1 and 4) and trace the new path taken
    from node 1 to node 5 (1, 2, 3, 5).

  \subsection{Experiment Results - No Failure}
  \subsubsection{Distance Vectors}
    \begin{verbatim}
    \end{verbatim}
  \subsubsection{Trace}
    \begin{verbatim}
    \end{verbatim}

  \subsection{Experiment Results - With Failure}
  \subsubsection{Distance Vectors}
    \begin{verbatim}
    \end{verbatim}
  \subsubsection{Trace}
    \begin{verbatim}
    \end{verbatim}

  \subsection{Analysis}
    In conclusion the distance-vector routing algorithm is simple to implement and performs a good job at managing failures amongst nodes.
    The main drawback to this approach is the time it takes for the nodes' distance vectors to converge on the best routes to take.
\end{document}
